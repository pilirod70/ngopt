% Template for PLoS
% Version 1.0 January 2009
%
% To compile to pdf, run:
% latex plos.template
% bibtex plos.template
% latex plos.template
% latex plos.template
% dvipdf plos.template

\documentclass[10pt]{article}

% amsmath package, useful for mathematical formulas
\usepackage{amsmath}
% amssymb package, useful for mathematical symbols
\usepackage{amssymb}

% graphicx package, useful for including eps and pdf graphics
% include graphics with the command \includegraphics
\usepackage{graphicx}

% cite package, to clean up citations in the main text. Do not remove.
\usepackage{cite}

\usepackage{color} 

% Use doublespacing - comment out for single spacing
%\usepackage{setspace} 
%\doublespacing


% Text layout
\topmargin 0.0cm
\oddsidemargin 0.5cm
\evensidemargin 0.5cm
\textwidth 16cm 
\textheight 21cm

% Bold the 'Figure #' in the caption and separate it with a period
% Captions will be left justified
\usepackage[labelfont=bf,labelsep=period,justification=raggedright]{caption}

% Use the PLoS provided bibtex style
\bibliographystyle{plos2009}

% Remove brackets from numbering in List of References
\makeatletter
\renewcommand{\@biblabel}[1]{\quad#1.}
\makeatother


\begin{document}

\renewcommand{\thesection}{S\arabic{section}}

\section{Supplementary Text}
\subsection*{Description of internal assembly pipeline parameters}

The A5 pipeline incorporates many algorithms, each of which require certain parameters to be set.
Table~\ref{tab:tab03} lists all of the parameters used in A5, along with how the value
of that parameter is calculated. We now discuss the effect of each of these parameter settings in turn.  
The SGA Quality Trim parameter sets a PHRED Q-score cutoff~\cite{Ewing1998} used for read trimming with the algorithm
implemented in BWA~\cite{bwa}. The SGA Quality Filter parameter sets a maximum number of bases
with a PHRED Q-score of below 3 that are permitted in a read before the read is discarded entirely.
The SGA Min Read Length sets the minimum allowed read length after quality trimming, reads shorter
than that will be discarded.

The IDBA min $k$-mer sets the starting $k$-mer size for \emph{de Bruijn} graph construction, and the max $k$-mer sets
the largest $k$ that will be used during graph simplification.

The SSPACE Overhang MinOverlap sets the number of nucleotides that a read must overlap with an existing
contig to be used for contig extension during scaffold gap filling. The SSPACE ExtendCall MinBases sets the minimum
number of reads covering a base that are required to call the base during contig extension. The SSPACE minimum
links sets the minimum number of read pairs that must be connecting a pair of contigs in order for them to be considered
for scaffolding, the reads must map to the region of the contig expected based on the insert size distribution.
For three contigs A, B, and C, where read pairs link A,B and A,C, the SSPACE Min Link Ratio sets the 
maximum allowable ratio between the number of links connecting A,C and A,B. If the ratio is below the threshold,
A,B will be scaffolded, otherwise A will not be scaffolded. Here A,B and A,C have the highest and 2$^{nd}$ highest number
of links between A and any other contig.

The SSPACE Merge MinOverlap sets the minimum number of bases that two scaffolded contigs must overlap in order to be merged
into a single contig.  The SSPACE Insert Mean is the mean value of the insert size for a paired-end or mate-pair library. 
The SSPACE Insert StDev is the standard deviation.

\bibliography{a5pipeline-appnote.plos.TextS1}

\end{document}
